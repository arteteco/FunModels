\def\bibliocommand{\bibliography{bibliography}}
\input{latBegin.txt}

\part{Introduction}
\label{introduction}

\chapter{Our closest cousins}
\label{ourclosestcousins}

The kingdom Fungi is one of the most diverse groups of organisms on Earth, and they are integral ecosystem agents with a huge impact on biogeochemical cycles, plant and animal pathology, plant nutrition and soil properties.
While historically clustered together with plants ~\citep{copeland1938, copeland1956}, towards the middle of last century it started to become clear that this lumping failed to properly deal with the differences between the two groups. In 1969 R. H. Whittaker published a paper dividing the organisms into five kingdoms: Animalia, Plantae, Fungi, Protista and Monera ~\citep{whittaker1969}. By the 70s this division became widely accepted, and the Kingdom Fungi was recognized.
Acknowledgement is just the first step in knowledge though. The understanding of the taxonomy, evolution and phylogenesis of fungi was still a matter of ample debate, one of those that may never end for lack of evidence. All the analysis were based on morphological differences, with all of its downsides.

Fossil record is very difficult to come by, as fungi do not biomineralize like animals do, and has proven not only inconclusive with regard to the origin of fungi, but also rather incomplete relative to the evolutionary history of the various fungal lineages. The earliest compendium of fossil fungi is from the late 19th century ~\citep{meschinelli1898}, and the relationship with plants in fossils was suggested around that period ~\citep{renault1896}, but despite the interest the difficulty in the interpretation of morphological data made it impossible to actually understand what happened.

Earliest fossil with the morphological features of a fungus is dated to around 1 billion years ago, and was found in the Arctic Canada ~\citep{loron2019}, and there is evidence of fungus-like organisms in fossils of around 800 Mya ~\citep{bonneville2020}. While those findings are rare, we surely have a richer diversity of fungi from the lower Devonian (around 400 Mya).

It wasn't until the large scale advent of molecular phylogenetics techniques that some light could be properly shed on the history of this group ~\citep{james2006}.
From molecular clock analysis seems like Fungi are sister group to animals, that is, the two lineages are close, diverging around 1.5 Billion years ago ~\citep{wang1999}. The two groups form one supergroup called Opisthokhonta ~\citep{cavalier-smith1987}, from the Greek opísthios (rear, posterior) and kontós (``pole'' i.e. ``flagellum''), since the group is characterized by flagellate cells that propel themselves with a single, posterior flagellum (in many cases lost).

IMG OF TREE WITH ANIMALS AND FUNGI AND PLANTS

\chapter{The ancient lovers}
\label{theancientlovers}

So: fungi are animals' cousins, and the two lineage diverged around 1.5 billion years ago. What happened then?

The ancestors of fungi are believed to be simple aquatic forms with flagellated spores, similar to members of the extant phylum Chytridiomycota (chytrids), which are now considered one of the early-diverging clade in the kingdom ~\citep{james2006}. The first terrestrial fungi colonized land probably before plants did ~\citep{heckman2001}, as saprobe (taking nutrition out of dead matter) and\slash or in symbiosis with organisms capable of photosynthesis.
It is commonly accepted that in order to colonize the land, plants had to develop a symbiotic relationship with fungi ~\citep{selosse1998, heckman2001, bonneville2020}, but it is not entirely clear whether this relationship was lichen-like or mycorrhizal-like.

IMG LICHEN

Lichens are the symbiotic relationship between a singol or more fungi (\emph{mycobiont})and a cyanobacteria or algae (\emph{photobyont}). Think about an algae floating in water: for many reasons, it is not really equipped to deal with the challenges of a terrestrial life style: mainly, it won't be able to mine substrate resources, to protect itself against dehydration, constant direct UV radiations and strong temperature fluctuations ~\citep{selosse1998, blackwell2000}. In a lichen, the photobiont is protected by the fungal stroma, and it can tolerate drought, cold, heat, intense light and barren rocky substrates. They also seem to be the first pioneers in a barren environment today, so everything would point to them being the right candidate for a first out-of-water plant-fungi symbiosis.

Yet, while this relationship evolved several times ~\citep{gargas1995}, the only phyla we know that are capable of such process (called \emph{lichenization}) are Ascomycota and, secondarly and later in time, Basidiomycota, and we can date the origin of those clades to about 400 Mya in the Devonian ~\citep{berbee1993}. Similarly we have fossils for lichens dating at the oldest in the Early Devonian (400 Mya) ~\citep{taylor1997, honegger2013}, while the first fossil land plants and fungi appeared 480 to 460 Mya, and molecular clock estimates suggests about 600--700 Mya ~\citep{berbee1993, heckman2001}.

Therefore, lichens were likely not what opened the way to plants for land colonization. Let's look now at a mycorrhizal-like relationship.

Mycorrhiza is the symbiotic association between plants and fungi happening in the rhizosphere, that is, the plant's root system. It consist in an exchange of resources between the fungus and the plant, ideally the plant providing sugar to the fungus and the fungus providing minerals and nutrients to the plant, even though it's hard to pinpoint who's benefiting who and such simplifications are only useful for educational purposes.

What we know is that fossils resembling mycorrhizal relationships have fossil evidence dating back to the Ordovician (with an age of about 460 million years), and are Glomales-like Arbuscular Mychorrhizal (AM hereafter) fungi, in a moment where the land flora is supposed to only consist of plants on the bryophitic level ~\citep{redecker2000}. Plants can photosynthetize. These fungi can extract minerals from the substrate with great efficiency, protect the root system, extend the range from which water can be taken and protect the plants from pathogens.
As in the worst TV series it's easy to see how this is going to end up in a passionate love story.

You could say ``\emph{wait a minute: those plants did not have true roots, how can we have a mycorrhizal relationship?}''. Good point. Fossil records provides evidence that fungal organisms entered in such symbiosis before the appearance of true roots, as long as there is a multicellular host for the fungus AM fungi are fine ~\citep{wang2006, bonfante2008}.

Whether as lichen or as mycorrhiza, the symbiosis between plants and fungi is one of the most important, most ancient relationship in the history of living beings and it surely played a crucial role in the successful colonization of the land by plants ~\citep{pirozynski1975, malloch1980, harley1987, trappe1987, selosse1998, brundrett2002}. The relationship is so beneficial (for one or both parts) that today is the norm, and is well established in c. 85\% of extant plants ~\citep{cairney2000, strullu-derrien2018}, with a high degree of complexity, diversification and situations ~\citep{heijden2015}

\chapter{Orchids}
\label{orchids}

Let's move the camera away from fungi for a second. Don't worry, we'll get back to them soon enough, but now we need to introduce the second protagonist of the present work: orchids.
Orchids are a diverse and widespread family of flowering plants, counting over 28,000 species in about 736 genera ~\citep{christenhusz2016}, second only to \emph{asteraceae} in terms of number even though they got on the scene only around 80 Mya ~\citep{ramirez2007}, not very long ago. They are cosmopolitan, with a distribution spanning all continents except Antarctica and including most major island groups

\begin{center}\rule{3in}{0.4pt}\end{center}

Next points:

\begin{itemize}
\item Conservation status of orchids

\item Ecology of orchids

\item Relationship with fungi

\end{itemize}

IMG ORCHIDS BY HAECKEL
and it shouldn't be a surprise: the benefits are great, often for both sides. Fungi help with the mineral intake, defending the plant against pathogen, alleviating salt stress ~\citep{evelin2009} and more, and

\input{latEnd.txt}

\end{document}
