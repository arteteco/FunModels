\input{latBegin.txt}

\part{Introduction}
\label{introduction}

The kingdom Fungi is one of the most diverse groups of organisms on Earth, and they are integral ecosystem agents with a huge impact on biogeochemical cycles, plant and animal pathology, plant nutrition and soil properties.
While historically clustered together with plants, towards the middle of last century it started to become clear that this lumping failed to properly deal with the differences between the two groups. In 1969 R. H. Whittaker published a paper dividing the organisms into five kingdoms: Animalia, Plantae, Fungi, Protista and Monera (Whittaker 1969). By the 70s this division became widely accepted, and the Kingdom Fungi was recognized.
Aknowledgement is not knowledge though. The understanding of the taxonomy, evolutionm and phylogenesis of fungi was still a matter of ample debate, one of those that may never end for lack of evidence. All the analysis were based on morphological differences, with all its downsides.

Fossil record is very difficult to come by, as fungi do not biomineralize like animals do, and has proven not only inconclusive with regard to the origin of fungi, but also rather incomplete relative to the evolutionary history of the various fungal lineages. The earliest compendium of fossil fungi is from the late 19th century (Meschinelli 1898), and the relationship with plants in fossils was suggested around that period (Renault 1896), but despite the interest the difficulty in the interpretation of morphological data made it impossible to actually understand what happened.

Earliest fossil with the morphological features of a fungus is dated to around 1 billion years ago, and was found in the Arctic Canada (Loron et al. 2019), while we have a richer diversity of fungi from the lower Devonian (around 400 Mya).

It wasn't until the large scale advent of molecular phylogenetics techniques that some light could be properly shed on the history of this group (James et al. 2006).
From molecular clock analysis, Fungi are sister group to animals, that is, the two lineages are close, diverging around 1.5 Billion years ago (D. Y.-C Wang et al. 1998). The two groups form one supergroup called Opisthokont (Cavalier-Smith T., 1987), from the Greek opísthios (rear, posterior) and kontós (``pole'' i.e. ``flagellum''), since the group is characterized by flagellate cells that propel themselves with a single, posterior flagellum (in many cases lost).

So: fungi are animals' cousins, and the two lineage diverged around 1.5 billion years ago. What happened then?

The ancestors of fungi are believed to be simple aquatic forms with flagellated spores, similar to members of the extant phylum Chytridiomycota (chytrids), which are now considered one of the early-diverging clade in the kingdom (James et al. 2006). The first terrestrial fungi colonized land long before plants did, as saprobe (taking nutrition out of dead matter) and in symbiosis with algae as well as with cyanobacteria, in one of the oldest, most important relationship in natural history: lichens.

\begin{center}\rule{3in}{0.4pt}\end{center}

The ancestsimple aquatic forms with flagellated spores, similar to members of the extant phylum Chytridiomycota (chytrids)

Most of the early fungi were aquatic, resembling what we may now call \emph{Chytridio} (even though this clade was found to be paraphyletic)

(Tedersoo et al. 2014)

\input{latEnd.txt}

\end{document}
